\chapter{Kesimpulan dan Saran}
\label{chap:simpulansaran}

Bab ini membahas tentang simpulan berdasarkan hasil dari analisis, implementasi, dan pengujian perangkat lunak yang telah dibuat, dan saran-saran untuk penelitian dan pengembangan selanjutnya.

\section{Kesimpulan}
\label{sec:simpulan}

Berdasarkan hasil dari analisis, implementasi, dan pengujian perangkat lunak Calcudoku yang telah dibuat, maka dapat diambil simpulan sebagai berkut:

\begin{enumerate}
\item Perangkat lunak permainan teka-teki Calcudoku dengan dua \textit{solver}, yaitu \textit{solver} dengan algoritma \textit{backtracking} dan \textit{solver} dengan algoritma \textit{hybrid genetic}, berhasil dibuat.

Perangkat lunak ini menerima input berupa soal teka-teki dan mampu menyelesaikan semua soal tersebut menggunakan algoritma \textit{backtracking}, dan sebagian soal tersebut (ukuran \textit{grid} \begin{math}4 \times 4\end{math} dan \begin{math}5 \times 5\end{math}) dengan menggunakan algoritma \textit{hybrid genetic}.

\item Algoritma \textit{backtracking} dapat menyelesaikan semua permainan yang diujikan. Pada ukuran \textit{grid} yang besar, algoritma \textit{backtracking} sangat lambat dalam menyelesaikan permainan.

Ada kemungkinan algoritma \textit{hybrid genetic} gagal dalam menyelesaikan permainan karena sifat acak dari algoritma \textit{hybrid genetic} ini. Semakin besar ukuran \textit{grid}, maka kemungkinan algoritma \textit{hybrid genetic} gagal dalam menyelesaikan permainan semakin besar.

Pada ukuran \textit{grid} yang kecil (\begin{math}4 \times 4\end{math} dan \begin{math}5 \times 5\end{math}), algoritma \textit{hybrid genetic} cenderung menyelesaikan permainan lebih lambat daripada algoritma \textit{backtracking}. Tetapi pada ukuran \textit{grid} yang besar (\begin{math}6 \times 6\end{math} ke atas), algoritma \textit{hybrid genetic} gagal dalam menyelesaikan permainan, sehingga performansinya tidak bisa dibandingkan dengan performansi algoritma \textit{backtracking}.

Semakin banyak sel yang diisi dalam tahap algoritma \textit{rule based}, semakin besar juga kemungkinan algoritma genetik untuk berhasil dalam menyelesaikan permainan dan semakin cepat juga algoritma genetik dalam menyelesaikan permainan.

Nilai dari parameter-parameter untuk algoritma genetik mempengaruhi kecepatan dan tingkat keberhasilan algoritma \textit{hybrid genetic} dalam menyelesaikan permainan. Semakin besar populasi dalam sebuah generasi sampai ke titik tertentu, dan semakin banyak generasi sampai ke titik tertentu, maka semakin besar juga kemungkinan algoritma \textit{hybrid genetic} berhasil dalam menyelesaikan permainan. Semakin besar tingkat \textit{elitism} dan tingkat mutasi sampai ke titik tertentu, maka semakin cepat juga algoritma \textit{hybrid genetic} dalam menyelesaikan permainan.
\end{enumerate}

\section{Saran}
\label{sec:saran}

Saran-saran yang dapat diberikan untuk mengembangkan penelitian ini adalah:

\begin{enumerate}
\item Memperbaiki GUI dari perangkat lunak ini agar petunjuk, yaitu angka tujuan dan operasi matematika yang ditentukan untuk sebuah \textit{cage}, dapat ditampilkan sebagaimana mestinya, yaitu pada di sudut kiri atas sel yang paling atas dan yang paling kiri dalam \textit{cage} tersebut.
\item Menambah aturan-aturan logika untuk algoritma \textit{rule based}, misalnya aturan \textit{naked subset} untuk \textit{cage} yang berukuran lebih besar dari 3 sel, aturan \textit{hidden subset} untuk \textit{cage} yang berukuran lebih besar dari 2 sel, aturan \textit{killer} combination untuk \textit{cage} yang berukuran lebih besar dari 2 sel, dan aturan \textit{evil twin} untuk \textit{cage} yang berukuran minimal 2 sel. Dengan menambah aturan-aturan logika untuk algoritma \textit{rule based}. Diharapkan, dengan menambah aturan-aturan logika untuk algoritma \textit{rule based}, maka tingkat kesuksesan algoritma \textit{hybrid genetic} dalam menyelesaikan permainan Calcudoku dapat meningkat.
\item Memperbaiki algoritma genetik, misalnya proses pemberian nilai kelayakan untuk kromosom, proses pemilihan kromosom untuk kawin silang dan mutasi, proses \textit{elitism}, proses kawin silang, dan proses mutasi, sehingga tingkat kesuksesan algoritma \textit{hybrid genetic} dalam menyelesaikan permainan Calcudoku dapat meningkat.
\end{enumerate}