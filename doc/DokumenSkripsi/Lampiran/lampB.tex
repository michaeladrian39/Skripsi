\chapter{Hasil Pengujian}
\label{chap:hasilpengujian}

Lampiran ini berisi hasil pengujian yang telah dilakukan.

\section{Algoritma \textit{Backtracking}}
\label{sec:hasilbt}

Berikut adalah hasil pengujian algoritma \textit{backtracking} untuk Calcudoku.

Hasil pengujian algoritma \textit{backtracking} untuk Calcudoku dengan ukuran \textit{grid} \begin{math}4 \times 4\end{math} dapat dilihat pada Tabel~\ref{tab:hasilbt4x4}.

\begin{table}
\centering
\captionsetup{justification=centering}
\caption[Hasil pengujian algoritma \textit{backtracking} untuk Calcudoku dengan ukuran \textit{grid} \protect\begin{math}4 \times 4\protect\end{math}]{Hasil pengujian algoritma \textit{backtracking} untuk Calcudoku dengan ukuran \textit{grid} \protect\begin{math}4 \times 4\protect\end{math}}
\begin{tabular}{| l | l |}
\hline
Nomor Soal & Kecepatan \\
\hline \hline
1 & 0.104 detik \\
\hline
2 & 0.008 detik \\
\hline
3 & 0.134 detik \\
\hline
4 & 0.082 detik \\
\hline
5 & 0.092 detik \\
\hline
6 & 0.008 detik \\
\hline
7 & 0.074 detik \\
\hline
8 & 0.185 detik \\
\hline
9 & 0.095 detik \\
\hline
10 & 0.106 detik \\
\hline
11 & 0.048 detik \\
\hline
12 & 0.122 detik \\
\hline
13 & 0.036 detik \\
\hline
14 & 0.043 detik \\
\hline
15 & 0.058 detik \\
\hline
16 & 0.013 detik \\
\hline
17 & 0.117 detik \\
\hline
18 & 0.076 detik \\
\hline
19 & 0.079 detik \\
\hline
20 & 0.043 detik \\
\hline
21 & 0.039 detik \\
\hline
22 & 0.055 detik \\
\hline
23 & 0.063 detik \\
\hline
24 & 0.048 detik \\
\hline
25 & 0.11 detik \\
\hline
26 & 0.036 detik \\
\hline
27 & 0.083 detik \\
\hline
28 & 0.059 detik \\
\hline
29 & 0.079 detik \\
\hline
30 & 0.058 detik \\
\hline
31 & 0.167 detik \\
\hline
32 & 0.046 detik \\
\hline
33 & 0.058 detik \\
\hline
34 & 0.021 detik \\
\hline
35 & 0.069 detik \\
\hline
36 & 0.024 detik \\
\hline
37 & 0.026 detik \\
\hline
38 & 0.017 detik \\
\hline
39 & 0.036 detik \\
\hline
\end{tabular}
\label{tab:hasilbt4x4}
\end{table}

Hasil pengujian algoritma \textit{backtracking} untuk Calcudoku dengan ukuran \textit{grid} \begin{math}5 \times 5\end{math} dapat dilihat pada Tabel~\ref{tab:hasilbt5x5}.

\begin{table}
\centering
\captionsetup{justification=centering}
\caption[Hasil pengujian algoritma \textit{backtracking} untuk Calcudoku dengan ukuran \textit{grid} \protect\begin{math}5 \times 5\protect\end{math}]{Hasil pengujian algoritma \textit{backtracking} untuk Calcudoku dengan ukuran \textit{grid} \protect\begin{math}5 \times 5\protect\end{math}}
\begin{tabular}{| l | l |}
\hline
Nomor Soal & Kecepatan \\
\hline \hline
1 & 0.257 detik \\
\hline
2 & 1.836 detik \\
\hline
3 & 0.958 detik \\
\hline
4 & 0.068 detik \\
\hline
5 & 0.816 detik \\
\hline
6 & 0.426 detik \\
\hline
7 & 1.17 detik \\
\hline
8 & 0.931 detik \\
\hline
9 & 1.017 detik \\
\hline
10 & 0.184 detik \\
\hline
11 & 0.716 detik \\
\hline
12 & 0.524 detik \\
\hline
13 & 0.15 detik \\
\hline
14 & 0.494 detik \\
\hline
15 & 0.438 detik \\
\hline
16 & 3.224 detik \\
\hline
17 & 0.276 detik \\
\hline
18 & 0.627 detik \\
\hline
19 & 1.755 detik \\
\hline
20 & 0.264 detik \\
\hline
21 & 0.446 detik \\
\hline
22 & 0.326 detik \\
\hline
23 & 0.092 detik \\
\hline
24 & 0.944 detik \\
\hline
25 & 0.137 detik \\
\hline
26 & 0.144 detik \\
\hline
\end{tabular}
\label{tab:hasilbt5x5}
\end{table}

Hasil pengujian algoritma \textit{backtracking} untuk Calcudoku dengan ukuran \textit{grid} \begin{math}6 \times 6\end{math} dapat dilihat pada Tabel~\ref{tab:hasilbt6x6}.

\begin{table}
\centering
\captionsetup{justification=centering}
\caption[Hasil pengujian algoritma \textit{backtracking} untuk Calcudoku dengan ukuran \textit{grid} \protect\begin{math}6 \times 6\protect\end{math}]{Hasil pengujian algoritma \textit{backtracking} untuk Calcudoku dengan ukuran \textit{grid} \protect\begin{math}6 \times 6\protect\end{math}}
\begin{tabular}{| l | l |}
\hline
Nomor Soal & Kecepatan \\
\hline \hline
1 & 6.315 detik \\
\hline
2 & 3.072 detik \\
\hline
3 & 2.306 detik \\
\hline
4 & 1.232 detik \\
\hline
5 & 0.775 detik \\
\hline
6 & 4.233 detik \\
\hline
7 & 2.498 detik \\
\hline
8 & 0.592 detik \\
\hline
9 & 5.529 detik \\
\hline
10 & 2.498 detik \\
\hline
11 & 0.62 detik \\
\hline
12 & 3.768 detik \\
\hline
13 & 22.784 detik \\
\hline
14 & 19.724 detik \\
\hline
15 & 0.866 detik \\
\hline
16 & 5.21 detik \\
\hline
17 & 2.327 detik \\
\hline
18 & 2.958 detik \\
\hline
19 & 5.97 detik \\
\hline
20 & 6.457 detik \\
\hline
21 & 4.011 detik \\
\hline
22 & 3.128 detik \\
\hline
23 & 243.767 detik \\
\hline
24 & 0.988 detik \\
\hline
25 & 0.172 detik \\
\hline
26 & 3.628 detik \\
\hline
27 & 8.873 detik \\
\hline
28 & 5.596 detik \\
\hline
29 & 1.1 detik \\
\hline
30 & 4.112 detik \\
\hline
31 & 1.328 detik \\
\hline
32 & 2.172 detik \\
\hline
33 & 5.381 detik \\
\hline
34 & 1.018 detik \\
\hline
35 & 72.546 detik \\
\hline
36 & 14.35 detik \\
\hline
37 & 14.662 detik \\
\hline
38 & 2.638 detik \\
\hline
39 & 50.561 detik \\
\hline
\end{tabular}
\label{tab:hasilbt6x6}
\end{table}

Hasil pengujian algoritma \textit{backtracking} untuk Calcudoku dengan ukuran \textit{grid} \begin{math}7 \times 7\end{math} dapat dilihat pada Tabel~\ref{tab:hasilbt7x7}.

\begin{table}
\centering
\captionsetup{justification=centering}
\caption[Hasil pengujian algoritma \textit{backtracking} untuk Calcudoku dengan ukuran \textit{grid} \protect\begin{math}7 \times 7\protect\end{math}]{Hasil pengujian algoritma \textit{backtracking} untuk Calcudoku dengan ukuran \textit{grid} \protect\begin{math}7 \times 7\protect\end{math}}
\begin{tabular}{| l | l |}
\hline
Nomor Soal & Kecepatan \\
\hline \hline
1 & 5924.967 detik \\
\hline
2 & 24.596 detik \\
\hline
3 & 40.597 detik \\
\hline
4 & 26.073 detik \\
\hline
5 & 75.227 detik \\
\hline
6 & 29.977 detik \\
\hline
7 & 338.317 detik \\
\hline
8 & 43.976 detik \\
\hline
9 & 109.051 detik \\
\hline
10 & 48.554 detik \\
\hline
11 & 43.503 detik \\
\hline
12 & 8.538 detik \\
\hline
13 & 1990.996 detik \\
\hline
14 & 64.485 detik \\
\hline
15 & 270.934 detik \\
\hline
\end{tabular}
\label{tab:hasilbt7x7}
\end{table}

Hasil pengujian algoritma \textit{backtracking} untuk Calcudoku dengan ukuran \textit{grid} \begin{math}8 \times 8\end{math} dapat dilihat pada Tabel~\ref{tab:hasilbt8x8}.

\begin{table}
\centering
\captionsetup{justification=centering}
\caption[Hasil pengujian algoritma \textit{backtracking} untuk Calcudoku dengan ukuran \textit{grid} \protect\begin{math}8 \times 8\protect\end{math}]{Hasil pengujian algoritma \textit{backtracking} untuk Calcudoku dengan ukuran \textit{grid} \protect\begin{math}8 \times 8\protect\end{math}}
\begin{tabular}{| l | l |}
\hline
Nomor Soal & Kecepatan \\
\hline \hline
1 & 1417.117 detik \\
\hline
2 & 4249.97 detik \\
\hline
3 & 2699.821 detik \\
\hline
4 & 180.779 detik \\
\hline
5 & 563.79 detik \\
\hline
6 & 6068.212 detik \\
\hline
7 & 1923.112 detik \\
\hline
8 & 727.159 detik \\
\hline
9 & 2817.854 detik \\
\hline
10 & 65.25 detik \\
\hline
11 & 4800.963 detik \\
\hline
12 & 1691.002 detik \\
\hline
13 & 545.492 detik \\
\hline
\end{tabular}
\label{tab:hasilbt8x8}
\end{table}

\clearpage

\section{Algoritma \textit{Hybrid Genetic}}
\label{sec:hasilhg}

Berikut adalah hasil pengujian algoritma \textit{hybrid genetic} untuk Calcudoku.

Pada semua skenario, algoritma \textit{hybrid genetic} gagal dalam menyelesaikan permainan dengan ukuran \textit{grid} \begin{math}6 \times 6\end{math} ke atas.

Hasil pengujian algoritma \textit{hybrid genetic} untuk Calcudoku dengan ukuran \textit{grid} \begin{math}4 \times 4\end{math} untuk Skenario 1 sampai dengan Skenario 4 dapat dilihat pada Tabel~\ref{tab:hasilhg4x41}.

\begin{table}
\centering
\captionsetup{justification=centering}
\caption[Hasil pengujian algoritma \textit{hybrid genetic} untuk Calcudoku dengan ukuran \textit{grid} \protect\begin{math}4 \times 4\protect\end{math} (Skenario 1-4)]{Hasil pengujian algoritma \textit{hybrid genetic} untuk Calcudoku dengan ukuran \textit{grid} \protect\begin{math}4 \times 4\protect\end{math} (Skenario 1-4)}
\begin{tabular}{| l | l | l | l | l |}
\hline
Nomor Soal & \makecell[c]{Kecepatan \\ Skenario 1} & \makecell[c]{Kecepatan \\ Skenario 2} & \makecell[c]{Kecepatan \\ Skenario 3} & \makecell[c]{Kecepatan \\ Skenario 4} \\
\hline \hline
1 & 9.801 detik & 11.025 detik & 10.413 detik & 11.637 detik \\
\hline
2 & 2.457 detik & 2.763 detik & 2.457 detik & 2.763 detik \\
\hline
3 & 9.801 detik & 11.025 detik & 10.413 detik & 11.637 detik \\
\hline
4 & 4.905 detik & 5.517 detik & 5.211 detik & 5.823 detik \\
\hline
5 & 0.314 detik & 0.314 detik & 0.314 detik & 0.314 detik \\
\hline
6 & 2.457 detik & 2.763 detik & 2.457 detik & 2.763 detik \\
\hline
7 & 4.905 detik & 5.517 detik & 5.211 detik & 5.823 detik \\
\hline
8 & 9.801 detik & 11.025 detik & 10.413 detik & 11.637 detik \\
\hline
9 & 0.314 detik & 0.314 detik & 0.314 detik & 0.314 detik \\
\hline
10 & 4.905 detik & 5.517 detik & 5.211 detik & 5.823 detik \\
\hline
11 & 4.905 detik & 5.517 detik & 5.211 detik & 5.823 detik \\
\hline
12 & 2.457 detik & 2.763 detik & 2.457 detik & 2.763 detik \\
\hline
13 & 2.457 detik & 2.763 detik & 2.457 detik & 2.763 detik \\
\hline
14 & 0.314 detik & 0.314 detik & 0.314 detik & 0.314 detik \\
\hline
15 & 9.801 detik & 11.025 detik & 10.413 detik & 11.637 detik \\
\hline
16 & 0.62 detik & 0.62 detik & 0.62 detik & 0.62 detik \\
\hline
17 & 1.232 detik & 1.232 detik & 1.232 detik & 1.232 detik \\
\hline
18 & 0.62 detik & 0.62 detik & 0.62 detik & 0.62 detik \\
\hline
19 & 4.905 detik & 5.517 detik & 5.211 detik & 5.823 detik \\
\hline
20 & 4.905 detik & 5.517 detik & 5.211 detik & 5.823 detik \\
\hline
21 & 2.457 detik & 2.763 detik & 2.457 detik & 2.763 detik \\
\hline
22 & 0.314 detik & 0.314 detik & 0.314 detik & 0.314 detik \\
\hline
23 & 9.801 detik & 11.025 detik & 10.413 detik & 11.637 detik \\
\hline
24 & 2.457 detik & 2.763 detik & 2.457 detik & 2.763 detik \\
\hline
25 & 0.008 detik & 0.008 detik & 0.008 detik & 0.008 detik \\
\hline
26 & 2.457 detik & 2.763 detik & 2.457 detik & 2.763 detik \\
\hline
27 & 4.905 detik & 5.517 detik & 5.211 detik & 5.823 detik \\
\hline
28 & 0.314 detik & 0.314 detik & 0.314 detik & 0.314 detik \\
\hline
29 & 2.457 detik & 2.763 detik & 2.457 detik & 2.763 detik \\
\hline
30 & 4.905 detik & 5.517 detik & 5.211 detik & 5.823 detik \\
\hline
31 & 9.801 detik & 11.025 detik & 10.413 detik & 11.637 detik \\
\hline
32 & 0.008 detik & 0.008 detik & 0.008 detik & 0.008 detik \\
\hline
33 & 9.801 detik & 11.025 detik & 10.413 detik & 11.637 detik \\
\hline
34 & 0.314 detik & 0.314 detik & 0.314 detik & 0.314 detik \\
\hline
35 & 1.232 detik & 1.232 detik & 1.232 detik & 1.232 detik \\
\hline
36 & 4.905 detik & 5.517 detik & 5.211 detik & 5.823 detik \\
\hline
37 & 2.457 detik & 2.763 detik & 2.457 detik & 2.763 detik \\
\hline
38 & 0.314 detik & 0.314 detik & 0.314 detik & 0.314 detik \\
\hline
39 & 4.905 detik & 5.517 detik & 5.211 detik & 5.823 detik \\
\hline
\end{tabular}
\label{tab:hasilhg4x41}
\end{table}

Hasil pengujian algoritma \textit{hybrid genetic} untuk Calcudoku dengan ukuran \textit{grid} \begin{math}4 \times 4\end{math} untuk Skenario 5 sampai dengan Skenario 8 dapat dilihat pada Tabel~\ref{tab:hasilhg4x42}.

\begin{table}
\centering
\captionsetup{justification=centering}
\caption[Hasil pengujian algoritma \textit{hybrid genetic} untuk Calcudoku dengan ukuran \textit{grid} \protect\begin{math}4 \times 4\protect\end{math} (Skenario 5-8)]{Hasil pengujian algoritma \textit{hybrid genetic} untuk Calcudoku dengan ukuran \textit{grid} \protect\begin{math}4 \times 4\protect\end{math} (Skenario 5-8)}
\begin{tabular}{| l | l | l | l | l |}
\hline
Nomor Soal & \makecell[c]{Kecepatan \\ Skenario 5} & \makecell[c]{Kecepatan \\ Skenario 6} & \makecell[c]{Kecepatan \\ Skenario 7} & \makecell[c]{Kecepatan \\ Skenario 8} \\
\hline \hline
1 & Gagal & Gagal & Gagal & Gagal \\
\hline
2 & 0.999 detik & 1.109 detik & 1.054 detik & 1.164 detik \\
\hline
3 & Gagal & Gagal & Gagal & Gagal \\
\hline
4 & Gagal & Gagal & Gagal & Gagal \\
\hline
5 & 0.063 detik & 0.063 detik & 0.063 detik & 0.063 detik \\
\hline
6 & 0.999 detik & 1.109 detik & 1.054 detik & 1.164 detik \\
\hline
7 & Gagal & Gagal & Gagal & Gagal \\
\hline
8 & Gagal & Gagal & Gagal & Gagal \\
\hline
9 & 0.036 detik & 0.036 detik & 0.036 detik & 0.036 detik \\
\hline
10 & Gagal & Gagal & Gagal & Gagal \\
\hline
11 & Gagal & Gagal & Gagal & Gagal \\
\hline
12 & 0.999 detik & 1.109 detik & 1.054 detik & 1.164 detik \\
\hline
13 & 0.999 detik & 1.109 detik & 1.054 detik & 1.164 detik \\
\hline
14 & 0.036 detik & 0.036 detik & 0.036 detik & 0.036 detik \\
\hline
15 & Gagal & Gagal & Gagal & Gagal \\
\hline
16 & 0.229 detik & 0.256 detik & 0.229 detik & 0.256 detik \\
\hline
17 & 0.339 detik & 0.394 detik & 0.366 detik & 0.421 detik \\
\hline
18 & 0.229 detik & 0.256 detik & 0.229 detik & 0.256 detik \\
\hline
19 & Gagal & Gagal & Gagal & Gagal \\
\hline
20 & Gagal & Gagal & Gagal & Gagal \\
\hline
21 & 0.999 detik & 1.109 detik & 1.054 detik & 1.164 detik \\
\hline
22 & 0.063 detik & 0.063 detik & 0.063 detik & 0.063 detik \\
\hline
23 & Gagal & Gagal & Gagal & Gagal \\
\hline
24 & 0.999 detik & 1.109 detik & 1.054 detik & 1.164 detik \\
\hline
25 & 0.008 detik & 0.008 detik & 0.008 detik & 0.008 detik \\
\hline
26 & 0.999 detik & 1.109 detik & 1.054 detik & 1.164 detik \\
\hline
27 & Gagal & Gagal & Gagal & Gagal \\
\hline
28 & 0.063 detik & 0.063 detik & 0.063 detik & 0.063 detik \\
\hline
29 & 0.999 detik & 1.109 detik & 1.054 detik & 1.164 detik \\
\hline
30 & Gagal & Gagal & Gagal & Gagal \\
\hline
31 & Gagal & Gagal & Gagal & Gagal \\
\hline
32 & 0.008 detik & 0.008 detik & 0.008 detik & 0.008 detik \\
\hline
33 & Gagal & Gagal & Gagal & Gagal \\
\hline
34 & 0.036 detik & 0.036 detik & 0.036 detik & 0.036 detik \\
\hline
35 & 0.339 detik & 0.394 detik & 0.366 detik & 0.421 detik \\
\hline
36 & Gagal & Gagal & Gagal & Gagal \\
\hline
37 & 0.999 detik & 1.109 detik & 1.054 detik & 1.164 detik \\
\hline
38 & 0.118 detik & 0.118 detik & 0.118 detik & 0.118 detik \\
\hline
39 & Gagal & Gagal & Gagal & Gagal \\
\hline
\end{tabular}
\label{tab:hasilhg4x42}
\end{table}

Hasil pengujian algoritma \textit{hybrid genetic} untuk Calcudoku dengan ukuran \textit{grid} \begin{math}4 \times 4\end{math} untuk Skenario 9 sampai dengan Skenario 12 dapat dilihat pada Tabel~\ref{tab:hasilhg4x43}.

\begin{table}
\centering
\captionsetup{justification=centering}
\caption[Hasil pengujian algoritma \textit{hybrid genetic} untuk Calcudoku dengan ukuran \textit{grid} \protect\begin{math}4 \times 4\protect\end{math} (Skenario 9-12)]{Hasil pengujian algoritma \textit{hybrid genetic} untuk Calcudoku dengan ukuran \textit{grid} \protect\begin{math}4 \times 4\protect\end{math} (Skenario 9-12)}
\begin{tabular}{| l | l | l | l | l |}
\hline
Nomor Soal & \makecell[c]{Kecepatan \\ Skenario 9} & \makecell[c]{Kecepatan \\ Skenario 10} & \makecell[c]{Kecepatan \\ Skenario 11} & \makecell[c]{Kecepatan \\ Skenario 12} \\
\hline \hline
1 & Gagal & Gagal & Gagal & Gagal \\
\hline
2 & Gagal & Gagal & Gagal & Gagal \\
\hline
3 & Gagal & Gagal & Gagal & Gagal \\
\hline
4 & Gagal & Gagal & Gagal & Gagal \\
\hline
5 & 0.314 detik & 0.314 detik & 0.314 detik & 0.314 detik \\
\hline
6 & Gagal & Gagal & Gagal & Gagal \\
\hline
7 & Gagal & Gagal & Gagal & Gagal \\
\hline
8 & Gagal & Gagal & Gagal & Gagal \\
\hline
9 & 0.314 detik & 0.314 detik & 0.314 detik & 0.314 detik \\
\hline
10 & Gagal & Gagal & Gagal & Gagal \\
\hline
11 & Gagal & Gagal & Gagal & Gagal \\
\hline
12 & Gagal & Gagal & Gagal & Gagal \\
\hline
13 & Gagal & Gagal & Gagal & Gagal \\
\hline
14 & 0.314 detik & 0.314 detik & 0.314 detik & 0.314 detik \\
\hline
15 & Gagal & Gagal & Gagal & Gagal \\
\hline
16 & 0.62 detik & 0.62 detik & 0.62 detik & 0.62 detik \\
\hline
17 & 1.232 detik & 1.232 detik & 1.232 detik & 1.232 detik \\
\hline
18 & 0.62 detik & 0.62 detik & 0.62 detik & 0.62 detik \\
\hline
19 & Gagal & Gagal & Gagal & Gagal \\
\hline
20 & Gagal & Gagal & Gagal & Gagal \\
\hline
21 & Gagal & Gagal & Gagal & Gagal \\
\hline
22 & 0.314 detik & 0.314 detik & 0.314 detik & 0.314 detik \\
\hline
23 & Gagal & Gagal & Gagal & Gagal \\
\hline
24 & Gagal & Gagal & Gagal & Gagal \\
\hline
25 & 0.008 detik & 0.008 detik & 0.008 detik & 0.008 detik \\
\hline
26 & Gagal & Gagal & Gagal & Gagal \\
\hline
27 & Gagal & Gagal & Gagal & Gagal \\
\hline
28 & 0.314 detik & 0.314 detik & 0.314 detik & 0.314 detik \\
\hline
29 & Gagal & Gagal & Gagal & Gagal \\
\hline
30 & Gagal & Gagal & Gagal & Gagal \\
\hline
31 & Gagal & Gagal & Gagal & Gagal \\
\hline
32 & 0.008 detik & 0.008 detik & 0.008 detik & 0.008 detik \\
\hline
33 & Gagal & Gagal & Gagal & Gagal \\
\hline
34 & 0.314 detik & 0.314 detik & 0.314 detik & 0.314 detik \\
\hline
35 & 1.232 detik & 1.232 detik & 1.232 detik & 1.232 detik \\
\hline
36 & Gagal & Gagal & Gagal & Gagal \\
\hline
37 & Gagal & Gagal & Gagal & Gagal \\
\hline
38 & 0.314 detik & 0.314 detik & 0.314 detik & 0.314 detik \\
\hline
39 & Gagal & Gagal & Gagal & Gagal \\
\hline
\end{tabular}
\label{tab:hasilhg4x43}
\end{table}

Hasil pengujian algoritma \textit{hybrid genetic} untuk Calcudoku dengan ukuran \textit{grid} \begin{math}4 \times 4\end{math} untuk Skenario 13 sampai dengan Skenario 16 dapat dilihat pada Tabel~\ref{tab:hasilhg4x44}.

\begin{table}
\centering
\captionsetup{justification=centering}
\caption[Hasil pengujian algoritma \textit{hybrid genetic} untuk Calcudoku dengan ukuran \textit{grid} \protect\begin{math}4 \times 4\protect\end{math} (Skenario 13-16)]{Hasil pengujian algoritma \textit{hybrid genetic} untuk Calcudoku dengan ukuran \textit{grid} \protect\begin{math}4 \times 4\protect\end{math} (Skenario 13-16)}
\begin{tabular}{| l | l | l | l | l |}
\hline
Nomor Soal & \makecell[c]{Kecepatan \\ Skenario 13} & \makecell[c]{Kecepatan \\ Skenario 14} & \makecell[c]{Kecepatan \\ Skenario 15} & \makecell[c]{Kecepatan \\ Skenario 16} \\
\hline \hline
1 & Gagal & Gagal & Gagal & Gagal \\
\hline
2 & Gagal & Gagal & Gagal & Gagal \\
\hline
3 & Gagal & Gagal & Gagal & Gagal \\
\hline
4 & Gagal & Gagal & Gagal & Gagal \\
\hline
5 & 0.063 detik & 0.063 detik & 0.063 detik & 0.063 detik \\
\hline
6 & Gagal & Gagal & Gagal & Gagal \\
\hline
7 & Gagal & Gagal & Gagal & Gagal \\
\hline
8 & Gagal & Gagal & Gagal & Gagal \\
\hline
9 & 0.036 detik & 0.036 detik & 0.036 detik & 0.036 detik \\
\hline
10 & Gagal & Gagal & Gagal & Gagal \\
\hline
11 & Gagal & Gagal & Gagal & Gagal \\
\hline
12 & Gagal & Gagal & Gagal & Gagal \\
\hline
13 & Gagal & Gagal & Gagal & Gagal \\
\hline
14 & 0.036 detik & 0.036 detik & 0.036 detik & 0.036 detik \\
\hline
15 & Gagal & Gagal & Gagal & Gagal \\
\hline
16 & Gagal & Gagal & Gagal & Gagal \\
\hline
17 & Gagal & Gagal & Gagal & Gagal \\
\hline
18 & Gagal & Gagal & Gagal & Gagal \\
\hline
19 & Gagal & Gagal & Gagal & Gagal \\
\hline
20 & Gagal & Gagal & Gagal & Gagal \\
\hline
21 & Gagal & Gagal & Gagal & Gagal \\
\hline
22 & 0.063 detik & 0.063 detik & 0.063 detik & 0.063 detik \\
\hline
23 & Gagal & Gagal & Gagal & Gagal \\
\hline
24 & Gagal & Gagal & Gagal & Gagal \\
\hline
25 & 0.008 detik & 0.008 detik & 0.008 detik & 0.008 detik \\
\hline
26 & Gagal & Gagal & Gagal & Gagal \\
\hline
27 & Gagal & Gagal & Gagal & Gagal \\
\hline
28 & 0.063 detik & 0.063 detik & 0.063 detik & 0.063 detik \\
\hline
29 & Gagal & Gagal & Gagal & Gagal \\
\hline
30 & Gagal & Gagal & Gagal & Gagal \\
\hline
31 & Gagal & Gagal & Gagal & Gagal \\
\hline
32 & 0.008 detik & 0.008 detik & 0.008 detik & 0.008 detik \\
\hline
33 & Gagal & Gagal & Gagal & Gagal \\
\hline
34 & 0.036 detik & 0.036 detik & 0.036 detik & 0.036 detik \\
\hline
35 & Gagal & Gagal & Gagal & Gagal \\
\hline
36 & Gagal & Gagal & Gagal & Gagal \\
\hline
37 & Gagal & Gagal & Gagal & Gagal \\
\hline
38 & 0.118 detik & 0.118 detik & 0.118 detik & 0.118 detik \\
\hline
39 & Gagal & Gagal & Gagal & Gagal \\
\hline
\end{tabular}
\label{tab:hasilhg4x44}
\end{table}

Hasil pengujian algoritma \textit{hybrid genetic} untuk Calcudoku dengan ukuran \textit{grid} \begin{math}5 \times 5\end{math} untuk Skenario 1 sampai dengan Skenario 4 dapat dilihat pada Tabel~\ref{tab:hasilhg5x51}.

\begin{table}
\centering
\captionsetup{justification=centering}
\caption[Hasil pengujian algoritma \textit{hybrid genetic} untuk Calcudoku dengan ukuran \textit{grid} \protect\begin{math}5 \times 5\protect\end{math} (Skenario 1-4)]{Hasil pengujian algoritma \textit{hybrid genetic} untuk Calcudoku dengan ukuran \textit{grid} \protect\begin{math}5 \times 5\protect\end{math} (Skenario 1-4)}
\begin{tabular}{| l | l | l | l | l |}
\hline
Nomor Soal & \makecell[c]{Kecepatan \\ Skenario 1} & \makecell[c]{Kecepatan \\ Skenario 2} & \makecell[c]{Kecepatan \\ Skenario 3} & \makecell[c]{Kecepatan \\ Skenario 4} \\
\hline \hline
1 & 18.369 detik & 19.899 detik & 19.134 detik & 20.664 detik \\
\hline
2 & Gagal & Gagal & Gagal & Gagal \\
\hline
3 & 0.391 detik & 0.391 detik & 0.391 detik & 0.391 detik \\
\hline
4 & Gagal & Gagal & Gagal & Gagal \\
\hline
5 & Gagal & Gagal & Gagal & Gagal \\
\hline
6 & Gagal & Gagal & Gagal & Gagal \\
\hline
7 & 12.249 detik & 13.779 detik & 13.014 detik & 14.544 detik \\
\hline
8 & Gagal & Gagal & Gagal & Gagal \\
\hline
9 & 12.249 detik & 13.779 detik & 13.014 detik & 14.544 detik \\
\hline
10 & Gagal & Gagal & Gagal & Gagal \\
\hline
11 & Gagal & Gagal & Gagal & Gagal \\
\hline
12 & Gagal & Gagal & Gagal & Gagal \\
\hline
13 & 0.391 detik & 0.391 detik & 0.391 detik & 0.391 detik \\
\hline
14 & Gagal & Gagal & Gagal & Gagal \\
\hline
15 & Gagal & Gagal & Gagal & Gagal \\
\hline
16 & Gagal & Gagal & Gagal & Gagal \\
\hline
17 & Gagal & Gagal & Gagal & Gagal \\
\hline
18 & 0.391 detik & 0.391 detik & 0.391 detik & 0.391 detik \\
\hline
19 & 18.369 detik & 19.899 detik & 19.134 detik & 20.664 detik \\
\hline
20 & Gagal & Gagal & Gagal & Gagal \\
\hline
21 & 4.599 detik & 4.981 detik & 4.599 detik & 4.981 detik \\
\hline
22 & 0.773 detik & 0.773 detik & 0.773 detik & 0.773 detik \\
\hline
23 & Gagal & Gagal & Gagal & Gagal \\
\hline
24 & 12.249 detik & 13.779 detik & 13.014 detik & 14.544 detik \\
\hline
25 & 12.249 detik & 13.779 detik & 13.014 detik & 14.544 detik \\
\hline
26 & Gagal & Gagal & Gagal & Gagal \\
\hline
\end{tabular}
\label{tab:hasilhg5x51}
\end{table}

Hasil pengujian algoritma \textit{hybrid genetic} untuk Calcudoku dengan ukuran \textit{grid} \begin{math}5 \times 5\end{math} untuk Skenario 5 sampai dengan Skenario 8 dapat dilihat pada Tabel~\ref{tab:hasilhg5x52}.

\begin{table}
\centering
\captionsetup{justification=centering}
\caption[Hasil pengujian algoritma \textit{hybrid genetic} untuk Calcudoku dengan ukuran \textit{grid} \protect\begin{math}5 \times 5\protect\end{math} (Skenario 5-8)]{Hasil pengujian algoritma \textit{hybrid genetic} untuk Calcudoku dengan ukuran \textit{grid} \protect\begin{math}5 \times 5\protect\end{math} (Skenario 5-8)}
\begin{tabular}{| l | l | l | l | l |}
\hline
Nomor Soal & \makecell[c]{Kecepatan \\ Skenario 5} & \makecell[c]{Kecepatan \\ Skenario 6} & \makecell[c]{Kecepatan \\ Skenario 7} & \makecell[c]{Kecepatan \\ Skenario 8} \\
\hline \hline
1 & Gagal & Gagal & Gagal & Gagal \\
\hline
2 & Gagal & Gagal & Gagal & Gagal \\
\hline
3 & 0.043 detik & 0.043 detik & 0.043 detik & 0.043 detik \\
\hline
4 & Gagal & Gagal & Gagal & Gagal \\
\hline
5 & Gagal & Gagal & Gagal & Gagal \\
\hline
6 & Gagal & Gagal & Gagal & Gagal \\
\hline
7 & Gagal & Gagal & Gagal & Gagal \\
\hline
8 & Gagal & Gagal & Gagal & Gagal \\
\hline
9 & Gagal & Gagal & Gagal & Gagal \\
\hline
10 & Gagal & Gagal & Gagal & Gagal \\
\hline
11 & Gagal & Gagal & Gagal & Gagal \\
\hline
12 & Gagal & Gagal & Gagal & Gagal \\
\hline
13 & 0.077 detik & 0.077 detik & 0.077 detik & 0.077 detik \\
\hline
14 & Gagal & Gagal & Gagal & Gagal \\
\hline
15 & Gagal & Gagal & Gagal & Gagal \\
\hline
16 & Gagal & Gagal & Gagal & Gagal \\
\hline
17 & Gagal & Gagal & Gagal & Gagal \\
\hline
18 & 0.043 detik & 0.043 detik & 0.043 detik & 0.043 detik \\
\hline
19 & Gagal & Gagal & Gagal & Gagal \\
\hline
20 & Gagal & Gagal & Gagal & Gagal \\
\hline
21 & 1.247 detik & 1.384 detik & 1.315 detik & 1.453 detik \\
\hline
22 & 0.146 detik & 0.146 detik & 0.146 detik & 0.146 detik \\
\hline
23 & Gagal & Gagal & Gagal & Gagal \\
\hline
24 & Gagal & Gagal & Gagal & Gagal \\
\hline
25 & Gagal & Gagal & Gagal & Gagal \\
\hline
26 & Gagal & Gagal & Gagal & Gagal \\
\hline
\end{tabular}
\label{tab:hasilhg5x52}
\end{table}

Hasil pengujian algoritma \textit{hybrid genetic} untuk Calcudoku dengan ukuran \textit{grid} \begin{math}5 \times 5\end{math} untuk Skenario 9 sampai dengan Skenario 12 dapat dilihat pada Tabel~\ref{tab:hasilhg5x53}.

\begin{table}
\centering
\captionsetup{justification=centering}
\caption[Hasil pengujian algoritma \textit{hybrid genetic} untuk Calcudoku dengan ukuran \textit{grid} \protect\begin{math}5 \times 5\protect\end{math} (Skenario 9-12)]{Hasil pengujian algoritma \textit{hybrid genetic} untuk Calcudoku dengan ukuran \textit{grid} \protect\begin{math}5 \times 5\protect\end{math} (Skenario 9-12)}
\begin{tabular}{| l | l | l | l | l |}
\hline
Nomor Soal & \makecell[c]{Kecepatan \\ Skenario 9} & \makecell[c]{Kecepatan \\ Skenario 10} & \makecell[c]{Kecepatan \\ Skenario 11} & \makecell[c]{Kecepatan \\ Skenario 12} \\
\hline \hline
1 & Gagal & Gagal & Gagal & Gagal \\
\hline
2 & Gagal & Gagal & Gagal & Gagal \\
\hline
3 & 0.391 detik & 0.391 detik & 0.391 detik & 0.391 detik \\
\hline
4 & Gagal & Gagal & Gagal & Gagal \\
\hline
5 & Gagal & Gagal & Gagal & Gagal \\
\hline
6 & Gagal & Gagal & Gagal & Gagal \\
\hline
7 & Gagal & Gagal & Gagal & Gagal \\
\hline
8 & Gagal & Gagal & Gagal & Gagal \\
\hline
9 & Gagal & Gagal & Gagal & Gagal \\
\hline
10 & Gagal & Gagal & Gagal & Gagal \\
\hline
11 & Gagal & Gagal & Gagal & Gagal \\
\hline
12 & Gagal & Gagal & Gagal & Gagal \\
\hline
13 & 0.391 detik & 0.391 detik & 0.391 detik & 0.391 detik \\
\hline
14 & Gagal & Gagal & Gagal & Gagal \\
\hline
15 & Gagal & Gagal & Gagal & Gagal \\
\hline
16 & Gagal & Gagal & Gagal & Gagal \\
\hline
17 & Gagal & Gagal & Gagal & Gagal \\
\hline
18 & 0.391 detik & 0.391 detik & 0.391 detik & 0.391 detik \\
\hline
19 & Gagal & Gagal & Gagal & Gagal \\
\hline
20 & Gagal & Gagal & Gagal & Gagal \\
\hline
21 & Gagal & Gagal & Gagal & Gagal \\
\hline
22 & 0.773 detik & 0.773 detik & 0.773 detik & 0.773 detik \\
\hline
23 & Gagal & Gagal & Gagal & Gagal \\
\hline
24 & Gagal & Gagal & Gagal & Gagal \\
\hline
25 & Gagal & Gagal & Gagal & Gagal \\
\hline
26 & Gagal & Gagal & Gagal & Gagal \\
\hline
\end{tabular}
\label{tab:hasilhg5x53}
\end{table}

Hasil pengujian algoritma \textit{hybrid genetic} untuk Calcudoku dengan ukuran \textit{grid} \begin{math}5 \times 5\end{math} untuk Skenario 13 sampai dengan Skenario 16 dapat dilihat pada Tabel~\ref{tab:hasilhg5x54}.

\begin{table}
\centering
\captionsetup{justification=centering}
\caption[Hasil pengujian algoritma \textit{hybrid genetic} untuk Calcudoku dengan ukuran \textit{grid} \protect\begin{math}5 \times 5\protect\end{math} (Skenario 13-16)]{Hasil pengujian algoritma \textit{hybrid genetic} untuk Calcudoku dengan ukuran \textit{grid} \protect\begin{math}5 \times 5\protect\end{math} (Skenario 13-16)}
\begin{tabular}{| l | l | l | l | l |}
\hline
Nomor Soal & \makecell[c]{Kecepatan \\ Skenario 13} & \makecell[c]{Kecepatan \\ Skenario 14} & \makecell[c]{Kecepatan \\ Skenario 15} & \makecell[c]{Kecepatan \\ Skenario 16} \\
\hline \hline
1 & Gagal & Gagal & Gagal & Gagal \\
\hline
2 & Gagal & Gagal & Gagal & Gagal \\
\hline
3 & 0.043 detik & 0.043 detik & 0.043 detik & 0.043 detik \\
\hline
4 & Gagal & Gagal & Gagal & Gagal \\
\hline
5 & Gagal & Gagal & Gagal & Gagal \\
\hline
6 & Gagal & Gagal & Gagal & Gagal \\
\hline
7 & Gagal & Gagal & Gagal & Gagal \\
\hline
8 & Gagal & Gagal & Gagal & Gagal \\
\hline
9 & Gagal & Gagal & Gagal & Gagal \\
\hline
10 & Gagal & Gagal & Gagal & Gagal \\
\hline
11 & Gagal & Gagal & Gagal & Gagal \\
\hline
12 & Gagal & Gagal & Gagal & Gagal \\
\hline
13 & 0.077 detik & 0.077 detik & 0.077 detik & 0.077 detik \\
\hline
14 & Gagal & Gagal & Gagal & Gagal \\
\hline
15 & Gagal & Gagal & Gagal & Gagal \\
\hline
16 & Gagal & Gagal & Gagal & Gagal \\
\hline
17 & Gagal & Gagal & Gagal & Gagal \\
\hline
18 & 0.043 detik & 0.043 detik & 0.043 detik & 0.043 detik \\
\hline
19 & Gagal & Gagal & Gagal & Gagal \\
\hline
20 & Gagal & Gagal & Gagal & Gagal \\
\hline
21 & Gagal & Gagal & Gagal & Gagal \\
\hline
22 & 0.146 detik & 0.146 detik & 0.146 detik & 0.146 detik \\
\hline
23 & Gagal & Gagal & Gagal & Gagal \\
\hline
24 & Gagal & Gagal & Gagal & Gagal \\
\hline
25 & Gagal & Gagal & Gagal & Gagal \\
\hline
26 & Gagal & Gagal & Gagal & Gagal \\
\hline
\end{tabular}
\label{tab:hasilhg5x54}
\end{table}

Daftar jumlah sel yang berhasil diisi oleh algoritma \textit{rule based} untuk Calcudoku dengan ukuran \textit{grid} \begin{math}4 \times 4\end{math} dapat dilihat pada Tabel~\ref{hasilhg4x45}.

\begin{table}
\centering
\captionsetup{justification=centering}
\caption[Daftar jumlah sel yang berhasil diisi oleh algoritma \textit{rule based} untuk Calcudoku dengan ukuran \textit{grid} \protect\begin{math}4 \times 4\protect\end{math}]{Daftar jumlah sel yang berhasil diisi oleh algoritma \textit{rule based} untuk Calcudoku dengan ukuran \textit{grid} \protect\begin{math}4 \times 4\protect\end{math}}
\begin{tabular}{| l | l |}
\hline
Nomor Soal & Jumlah Sel Diisi Algoritma \textit{Rule Based} \\
\hline \hline
1 & 0 \\
\hline
2 & 2 \\
\hline
3 & 0 \\
\hline
4 & 1 \\
\hline
5 & 6 \\
\hline
6 & 2 \\
\hline
7 & 1 \\
\hline
8 & 0 \\
\hline
9 & 9 \\
\hline
10 & 1 \\
\hline
11 & 1 \\
\hline
12 & 2 \\
\hline
13 & 2 \\
\hline
14 & 8 \\
\hline
15 & 0 \\
\hline
16 & 4 \\
\hline
17 & 3 \\
\hline
18 & 4 \\
\hline
19 & 1 \\
\hline
20 & 1 \\
\hline
21 & 2 \\
\hline
22 & 6 \\
\hline
23 & 0 \\
\hline
24 & 2 \\
\hline
25 & 16 \\
\hline
26 & 2 \\
\hline
27 & 1 \\
\hline
28 & 6 \\
\hline
29 & 2 \\
\hline
30 & 1 \\
\hline
31 & 0 \\
\hline
32 & 16 \\
\hline
33 & 0 \\
\hline
34 & 8 \\
\hline
35 & 3 \\
\hline
36 & 1 \\
\hline
37 & 2 \\
\hline
38 & 5 \\
\hline
39 & 1 \\
\hline
\end{tabular}
\label{tab:hasilhg4x45}
\end{table}

Daftar jumlah sel yang berhasil diisi oleh algoritma \textit{rule based} untuk Calcudoku dengan ukuran \textit{grid} \begin{math}5 \times 5\end{math} dapat dilihat pada Tabel~\ref{hasilhg5x55}.

\begin{table}
\centering
\captionsetup{justification=centering}
\caption[Daftar jumlah sel yang berhasil diisi oleh algoritma \textit{rule based} untuk Calcudoku dengan ukuran \textit{grid} \protect\begin{math}5 \times 5\protect\end{math}]{Daftar jumlah sel yang berhasil diisi oleh algoritma \textit{rule based} untuk Calcudoku dengan ukuran \textit{grid} \protect\begin{math}5 \times 5\protect\end{math}}
\begin{tabular}{| l | l |}
\hline
Nomor Soal & Jumlah Sel Diisi Algoritma \textit{Rule Based} \\
\hline \hline
1 & 4 \\
\hline
2 & 0 \\
\hline
3 & 15 \\
\hline
4 & 1 \\
\hline
5 & 1 \\
\hline
6 & 1 \\
\hline
7 & 5 \\
\hline
8 & 1 \\
\hline
9 & 5 \\
\hline
10 & 0 \\
\hline
11 & 0 \\
\hline
12 & 3 \\
\hline
13 & 15 \\
\hline
14 & 2 \\
\hline
15 & 0 \\
\hline
16 & 1 \\
\hline
17 & 0 \\
\hline
18 & 16 \\
\hline
19 & 4 \\
\hline
20 & 2 \\
\hline
21 & 8 \\
\hline
22 & 14 \\
\hline
23 & 1 \\
\hline
24 & 5 \\
\hline
25 & 5 \\
\hline
26 & 1 \\
\hline
\end{tabular}
\label{tab:hasilhg5x55}
\end{table}