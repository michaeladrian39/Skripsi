%_____________________________________________________________________________
%=============================================================================
% data.tex v8 (02-10-2016) \ldots dibuat oleh Lionov - Informatika FTIS UNPAR
%
% Perubahan pada versi 8 (02-10-2016)
%	- Perubahan keterangan pada spacing: Otomatis spasi 1 untuk buku skripsi 
%	  final dan 1.5 untuk buku sidang
%	- Penggunaan kantlipsum
%_____________________________________________________________________________
%=============================================================================

%=============================================================================
% 								PETUNJUK
%=============================================================================
% Ini adalah file data (data.tex)
% Masukkan ke dalam file ini, data-data yang diperlukan oleh template ini
% Cara memasukkan data dijelaskan di setiap bagian
% Data yang WAJIB dan HARUS diisi dengan baik dan benar adalah SELURUHNYA !!
% Hilangkan tanda << dan >> jika anda menemukannya
%=============================================================================

%_____________________________________________________________________________
%=============================================================================
% 								BAGIAN 0
%=============================================================================
% PERHATIAN!! PERHATIAN!! Bagian ini hanya ada untuk sementara saja
% Jika "DAFTAR ISI" tidak bisa berada di bagian tengah halaman, isi dengan XXX
% jika sudah benar posisinya, biarkan kosong (i.e. \daftarIsiError{ })
%=============================================================================
\daftarIsiError{ }
%=============================================================================

%_____________________________________________________________________________
%=============================================================================
% 								BAGIAN I
%=============================================================================
% Tambahkan package2 lain yang anda butuhkan di sini
%=============================================================================
\usepackage{booktabs} 
\usepackage[table]{xcolor}
\usepackage{longtable}
\usepackage{amssymb}
\usepackage{todo}
\usepackage{verbatim} 		%multilne comment
\usepackage{pgfplots}
%=============================================================================

%_____________________________________________________________________________
%=============================================================================
% 								BAGIAN II
%=============================================================================
% Mode dokumen: menetukan halaman depan dari dokumen, apakah harus mengandung 
% prakata/pernyataan/abstrak dll (termasuk daftar gambar/tabel/isi) ?
% - kosong : tidak ada halaman depan sama sekali (untuk dokumen yang 
%            dipergunakan pada proses bimbingan)
% - cover : cover saja tanpa daftar isi, gambar dan tabel
% - sidang : cover, daftar isi, gambar, tabel 
% - sidang_akhir : mode sidang + abstrak + abstract
% - final : seluruh halaman awal dokumen (untuk cetak final)
% Jika tidak ingin mencetak daftar tabel/gambar (misalkan karena tidak ada 
% isinya), edit manual di baris 439 dan 440 pada file main.tex
%=============================================================================
% \mode{kosong}
% \mode{cover}
% \mode{sidang}
% \mode{sidang_akhir}
 \mode{final} 
%=============================================================================

%_____________________________________________________________________________
%=============================================================================
% 								BAGIAN III
%=============================================================================
% Line numbering: penomoran setiap baris, otomatis di-reset setiap berganti
% halaman
% - yes: setiap baris diberi nomor
% - no : baris tidak diberi nomor, otomatis untuk mode final
%=============================================================================
\linenumber{no}
%=============================================================================

%_____________________________________________________________________________
%=============================================================================
% 								BAGIAN IV
%=============================================================================
% Linespacing: jarak antara baris 
% - single	: wajib (dan otomatis jika ingin mencetak buku skripsi, opsi yang 
%			  disediakan untuk bimbingan, jika pembimbing tidak keberatan 
%			  (untuk menghemat kertas)
% - onehalf	: default dan wajib (dan otomatis) jika ingin mencetak dokumen
%             untuk sidang.
% - double 	: jarak yang lebih lebar lagi, jika pembimbing berniat memberi 
%             catatan yg banyak di antara baris (dianjurkan untuk bimbingan)
%=============================================================================
\linespacing{single}
%\linespacing{onehalf}
%\linespacing{double}
%=============================================================================

%_____________________________________________________________________________
%=============================================================================
% 								BAGIAN V
%=============================================================================
% Tidak semua skripsi memuat gambar dan/atau tabel. Untuk skripsi yang seperti
% itu, tidak diperlukan Daftar Gambar dan Daftar Tabel. Sayangnya hal ini 
% sulit dilakukan secara manual karena membutuhkan kedisiplinan pengguna 
% template.  
% Jika tidak akan menampilkan Daftar Gambar/Tabel, isi dengan NO. Jika ingin
% menampilkan, kosongkan parameter (i.e. \gambar{ }, \tabel{ })
%=============================================================================
\gambar{ }
\tabel{ }
%=============================================================================

%_____________________________________________________________________________
%=============================================================================
% 								BAGIAN VI
%=============================================================================
% Bab yang akan dicetak: isi dengan angka 1,2,3 s.d 9, sehingga bisa digunakan
% untuk mencetak hanya 1 atau beberapa bab saja
% Jika lebih dari 1 bab, pisahkan dengan ',', bab akan dicetak terurut sesuai 
% urutan bab (e.g. \bab{1,2,3}).
% Untuk mencetak seluruh bab, kosongkan parameter (i.e. \bab{ })  
% Catatan: Jika ingin menambahkan bab ke-10 dan seterusnya, harus dilakukan 
% secara manual
%=============================================================================
\bab{ }
%=============================================================================

%_____________________________________________________________________________
%=============================================================================
% 								BAGIAN VII
%=============================================================================
% Lampiran yang akan dicetak: isi dengan huruf A,B,C s.d I, sehingga bisa 
% digunakan untuk mencetak hanya 1 atau beberapa lampiran saja
% Jika lebih dari 1 lampiran, pisahkan dengan ',', lampiran akan dicetak 
% terurut sesuai urutan lampiran (e.g. \bab{A,B,C}).
% Jika tidak ingin mencetak lampiran apapun, isi dengan -1 (i.e. \lampiran{-1})
% Untuk mencetak seluruh mapiran, kosongkan parameter (i.e. \lampiran{ })  
% Catatan: Jika ingin menambahkan lampiran ke-J dan seterusnya, harus 
% dilakukan secara manual
%=============================================================================
\lampiran{ }
%=============================================================================

%_____________________________________________________________________________
%=============================================================================
% 								BAGIAN VIII
%=============================================================================
% Data diri dan skripsi/tugas akhir
% - namanpm: Nama dan NPM anda, penggunaan huruf besar untuk nama harus benar
%			 dan gunakan 10 digit npm UNPAR, PASTIKAN BAHWA BENAR !!!
%			 (e.g. \namanpm{Jane Doe}{1992710001}
% - judul : Dalam bahasa Indonesia, perhatikan penggunaan huruf besar, judul
%			tidak menggunakan huruf besar seluruhnya !!! 
% - tanggal : isi dengan {tangga}{bulan}{tahun} dalam angka numerik, jangan 
%			  menuliskan kata (e.g. AGUSTUS) dalam isian bulan
%			  Tanggal ini adalah tanggal dimana anda akan melaksanakan sidang 
%			  ujian akhir skripsi/tugas akhir
% - pembimbing: isi dengan pembimbing anda, lihat daftar dosen di file dosen.tex
%				jika pembimbing hanya 1, kosongkan parameter kedua 
%				(e.g. \pembimbing{\JND}{  } ) , \JND adalah kode dosen
% - penguji : isi dengan para penguji anda, lihat daftar dosen di file dosen.tex
%				(e.g. \penguji{\JHD}{\JCD} ) , \JND dan \JCD adalah kode dosen
% !!Lihat singkatan pembimbing dan penguji anda di file dosen.tex
%=============================================================================
\namanpm{Michael Adrian}{2013730039}	%hilangkan tanda << & >>
\tanggal{<<tanggal>>}{<<bulan>>}{2017}				%hilangkan tanda << & >>
\pembimbing{\CEN}{<<pembimbing 2>>} %hilangkan tanda << & >>
\penguji{<<penguji 1>>}{<<penguji 2>>} 				%hilangkan tanda << & >>
%=============================================================================

%_____________________________________________________________________________
%=============================================================================
% 								BAGIAN IX
%=============================================================================
% Judul dan title : judul bhs indonesia dan inggris
% - judulINA: judul dalam bahasa indonesia
% - judulENG: title in english
% PERHATIAN: - langsung mulai setelah '{' awal, jangan mulai menulis di baris 
%			   bawahnya
%			 - Gunakan \texorpdfstring{\\}{} untuk pindah ke baris baru
%			 - Judul TIDAK ditulis dengan menggunakan huruf besar seluruhnya !!
%			 - Gunakan perintah \texorpdfstring{\\}{} untuk baris baru
%=============================================================================
\judulINA{Perbandingan Algoritma Backtracking dengan Algoritma Hybrid Genetic untuk Menyelesaikan Permainan Calcudoku}
\judulENG{Comparison of the Backtracking Algorithm and the Hybrid Genetic Algorithm to Solve the Calcudoku Puzzle}
%_____________________________________________________________________________
%=============================================================================
% 								BAGIAN X
%=============================================================================
% Abstrak dan abstract : abstrak bhs indonesia dan inggris
% - abstrakINA: abstrak bahasa indonesia
% - abstrakENG: abstract in english
% PERHATIAN: langsung mulai setelah '{' awal, jangan mulai menulis di baris 
%			 bawahnya
%=============================================================================
\abstrakINA
{Calcudoku adalah sebuah permainan teka-teki angka. Tujuan dari teka-teki ini adalah mengisi setiap sel dalam \textit{grid} dengan angka 1 sampai \begin{math}n\end{math} tanpa pengulangan angka dalam setiap kolomnya dan barisnya untuk \textit{grid} berukuran \begin{math}n \times n\end{math}. \textit{Grid} ini dibagi menjadi sejumlah \textit{cage} dengan setiap \textit{cage} yang jumlah selnya bervariasi. Setiap \textit{cage} dibatasi oleh garis yang lebih tebal daripada garis pembatas antar sel. Angka-angka dalam satu \textit{cage} yang sama harus menghasilkan angka tujuan yang telah ditentukan jika dihitung menggunakan operasi matematika yang ditentukan. Angka-angka dalam satu \textit{cage} juga boleh berulang, selama pengulangan tidak terjadi dalam satu kolom atau baris yang sama.

Dua algoritma telah terbukti berhasil dalam menyelesaikan Calcudoku, yaitu algoritma \textit{backtracking} dan algoritma \textit{hybrid genetic}.

Algoritma \textit{backtracking} adalah sebuah algoritma umum yang mencari solusi dengan mencoba salah satu dari beberapa pilihan, jika pilihan yang dipilih ternyata salah, komputasi dimulai lagi pada titik pilihan dan mencoba pilihan lainnya.

Algoritma \textit{hybrid genetic} dalam kasus ini adalah gabungan dari algoritma \textit{rule based} dan algoritma genetik. Algoritma \textit{rule based} adalah sebuah algoritma berbasis aturan logika untuk menyelesaikan Calcudoku. Algoritma genetik adalah salah satu teknik heuristik \textit{Generate and Test} yang terinspirasi oleh sistem seleksi alam yang mencari solusi dengan menggunakan operator-operator genetik seperti mutasi, kawin silang, dan \textit{elitism}.

Hasil dari penelitian ini adalah perangkat lunak algoritma \textit{backtracking} dapat menyelesaikan semua permainan yang diujikan, tetapi pada ukuran \textit{grid} yang besar, algoritma \textit{backtracking} sangat lambat dalam menyelesaikan permainan, ada kemungkinan algoritma \textit{hybrid genetic} gagal dalam menyelesaikan permainan karena sifat acak dari algoritma \textit{hybrid genetic} ini, semakin besar ukuran \textit{grid}, maka kemungkinan algoritma \textit{hybrid genetic} gagal dalam menyelesaikan permainan semakin besar, pada ukuran \textit{grid} yang kecil, algoritma \textit{hybrid genetic} cenderung menyelesaikan permainan lebih lambat daripada algoritma \textit{backtracking}, tetapi pada ukuran \textit{grid} yang besar, algoritma \textit{hybrid genetic} mungkin mampu menyelesaikan permainan lebih cepat daripada algoritma \textit{backtracking}, tetapi hal ini tidak dapat dibuktikan karena algoritma \textit{hybrid genetic} gagal dalam menyelesaikan permainan dengan ukuran \textit{grid} yang besar, dan nilai untuk parameter-parameter algoritma genetik mempengaruhi kecepatan dan tingkat keberhasilan algoritma \textit{hybrid genetic} dalam menyelesaikan permainan.
}
\abstrakENG
{Calcudoku is a number puzzle game. The goal of this puzzle is to fill each cell in the grid with the numbers from 1 to n without repetitions of the numbers in each column and row for grid with the size of n x n. The grid is divided into a number of cages, with each cage contains a variable number of cells. Each cage is bordered with a thicker line than cell border line. Numbers in the same cage must produced the predetermined target number if calculated using the predetermined mathematical operatin. Numbers in a cage can be repeated, as long as the repetitions do not occur in the same column or row.

Two algorithms have been proven to successfully solve Calcudoku. The two algorithms are the backtracking algorithm and the hybrid genetic algorithm.

The backtracking algorithm is a general algorithm with finds a solution by trying one of several choices, if the choice proves to be incorrect, the computation restarts at the point of choice and tries another choice.

In this case, the hybrid genetic algorithm is a combination of the rule based algorithm and the genetic algorithm. Rule based algorithm uses logical rules to solve Calcudoku. Genetic algorithm is a heuristic technique inspired by the process of natural selection, which tries to find a solution by relying on genetic operators such as mutation, crossover, and elitism.

The result of this research is that the backtracking algorithm solved all puzzles, but on large grids, the algorithm is very slow in solving the puzzle, there is a chance that the hybrid genetic algorithm failed in solving the puzzle due to the random nature of the algorithm, the larger the grid, the higher the chance that the algorithm will fail in solving the puzzle, on small grids, the hybrid genetic algorithm tends to solve the puzzle slower than the backtracking algorithm, but on larger grids, the hybrid genetic algorithm may be able to solve the puzzle faster than the backtracking algorithm, but this cannot be proven because the hybrid genetic algorithm failed in solving puzzles with large grids, and the values of the genetic algorithm parameters influences the speed and the success rate of the hybrid genetic algorithm in solving the puzzle.}
%=============================================================================

%_____________________________________________________________________________
%=============================================================================
% 								BAGIAN XI
%=============================================================================
% Kata-kata kunci dan keywords : diletakkan di bawah abstrak (ina dan eng)
% - kunciINA: kata-kata kunci dalam bahasa indonesia
% - kunciENG: keywords in english
%=============================================================================
\kunciINA{Calcudoku, algoritma \textit{backtracking}, algoritma \textit{hybrid genetic}, algoritma \textit{rule based}, \textit{algoritma genetik}}
\kunciENG{Calcudoku, backtracking algorithm, hybrid genetic algorithm, rule based algorithm, genetic algorithm}
%=============================================================================

%_____________________________________________________________________________
%=============================================================================
% 								BAGIAN XII
%=============================================================================
% Persembahan : kepada siapa anda mempersembahkan skripsi ini ...
%=============================================================================
\untuk{Lorem ipsum dolor sit amet, consectetur adipiscing elit.}
%=============================================================================

%_____________________________________________________________________________
%=============================================================================
% 								BAGIAN XIII
%=============================================================================
% Kata Pengantar: tempat anda menuliskan kata pengantar dan ucapan terima 
% kasih kepada yang telah membantu anda bla bla bla ....  
%=============================================================================
\prakata
{Lorem ipsum dolor sit amet, consectetuer adipiscing elit, sed diam 
nonummy nibh euismod tincidunt ut laoreet dolore magna aliquam erat 
volutpat. Ut wisi enim ad minim veniam, quis nostrud exerci tation 
ullamcorper suscipit lobortis nisl ut aliquip ex ea commodo consequat. 
Duis autem vel eum iriure dolor in hendrerit in vulputate velit esse 
molestie consequat, vel illum dolore eu feugiat nulla facilisis at vero 
eros et accumsan et iusto odio dignissim qui blandit praesent luptatum 
zzril delenit augue duis dolore te feugait nulla facilisi. Nam liber 
tempor cum soluta nobis eleifend option congue nihil imperdiet doming 
id quod mazim placerat facer possim assum. Typi non habent claritatem 
insitam; est usus legentis in iis qui facit eorum claritatem. 
Investigationes demonstraverunt lectores legere me lius quod ii legunt 
saepius. Claritas est etiam processus dynamicus, qui sequitur 
mutationem consuetudium lectorum. Mirum est notare quam littera 
gothica, quam nunc putamus parum claram, anteposuerit litterarum formas 
humanitatis per seacula quarta decima et quinta decima. Eodem modo 
typi, qui nunc nobis videntur parum clari, fiant sollemnes in futurum.

Lorem ipsum dolor sit amet, consectetur adipiscing elit, sed do eiusmod 
tempor incididunt ut labore et dolore magna aliqua. Ut enim ad minim 
veniam, quis nostrud exercitation ullamco laboris nisi ut aliquip ex ea 
commodo consequat. Duis aute irure dolor in reprehenderit in voluptate 
velit esse cillum dolore eu fugiat nulla pariatur. Excepteur sint 
occaecat cupidatat non proident, sunt in culpa qui officia deserunt 
mollit anim id est laborum.}
%=============================================================================

%_____________________________________________________________________________
%=============================================================================
% 								BAGIAN XIV
%=============================================================================
% Tambahkan hyphen (pemenggalan kata) yang anda butuhkan di sini 
%=============================================================================
\hyphenation{ma-te-ma-ti-ka}
\hyphenation{fi-si-ka}
\hyphenation{tek-nik}
\hyphenation{in-for-ma-ti-ka}
\hyphenation{cal-cu-do-ku}
\hyphenation{al-go-rit-ma}
\hyphenation{back-track-ing}
\hyphenation{hy-brid}
\hyphenation{ge-ne-tic}
%=============================================================================

%_____________________________________________________________________________
%=============================================================================
% 								BAGIAN XV
%=============================================================================
% Tambahkan perintah yang anda buat sendiri di sini 
%=============================================================================
\newcommand{\vtemplateauthor}{lionov}
\pgfplotsset{compat=newest}
\usetikzlibrary{patterns}
%=============================================================================

% Copyright \textcopyright [Lionov] [09-10-2016]. All rights reserved